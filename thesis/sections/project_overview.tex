The main goal for this project is to have a set of \abcpt{}s. Each tool deals with a specific
problem and tries to solve it in a simple and efficient way. In order to simplify the creation of an
\abcpt{}, a rule-based \ac{DSL} - \abcdt{} - was designed. From \abcdt{}'s rules an \abcpt{} is
obtained whose main algorithm follows a traditional compiler architecture.

\subsection*{Design Goals}

A set of design goals was defined to guide the project's implementation.

\subsubsection*{Toolkit}
% Problems associated with the processing of textual musical notation:
% 
% \begin{description}
%   \item \textbf{Automatic validation of the notation and error detection} \hfill\\
%     When writing \abc{}, since it is \ac{ASCII} based, it is relatively easy to write errors on the score.
%     One of the features that is required for this toolkit is for it to be able to detect those
%     errors thus validating the notation in terms of syntax, lexicon, among others that are listed
%     below:
%     \begin{itemize}
%       \item \textbf{syntax}\\
%         e.g.: beats per measure, different key definitions in voices;
%       \item \textbf{lexicon}\\
%         e.g.: inappropriate use of symbols;
%       \item \textbf{misalignment}\\
%         e.g.: voice disparity due to transcription errors;
%       \item \textbf{suspicious use of musical symbols}\\
%         e.g.: use of chords out of context in a certain musical style;
%         e.g.: use of notes out of vocal range;
%     \end{itemize}
%   \item \textbf{Error fixing} \hfill\\
%     Since it can detect erros it's only natural that they can be fixed, either by automatically
%     setting the appropriate fix or by suggesting one to the user. Typically the errors fixed will be
%     of syntactic or lexical nature.
%     \begin{itemize}
%       \item \textbf{syntactic}\\
%         e.g.: when there are many consecutive rests, then a suggestion is made to join them by
%         measure;
%         e.g.: when the key is changed in only one voice, then that change is reproduced in all %         voices;
%       \item \textbf{lexical}\\
%         e.g.: when there's an orthographic or a capitalization error in reserved words, then it is
%         automatically fixed;
%     \end{itemize}
% \end{description}

Each \abcpt{} should:

\begin{itemize}
  \item \textbf{Deal with real \abc{} music} \hfill\\
    Each tool should be able to deal with more than just a sequence of notes. Musical elements other
    than notes and rests, like lyrics or accompaniment chords, are parsed and can be processed.
    Also, unexpected elements shouldn't break the tool's process.
  \item \textbf{Follow the \unix{} philosophy} \hfill\\
    Each tool tackles a single problem and can be composed with others to solve more complex
    problems.
  \item \textbf{Be open-source} \hfill\\
    The source code will be available to the general public therefore it will follow community
    practices and will be installable and usable as a third-party tool. The existence of an
    open-source community allows the exchange of information and ideas between developers and users.
    Interesting things can come out of a discussion in this mean, like new features or a solution to
    a certain problem.
\end{itemize}

\subsubsection*{ABC::DT}

\abcdt{} is a rule-based \ac{DSL} which aims to help the creation of \abcpt{}s in a simple and
compact way. Therefore, in order to achieve that simpleness it must have the following features:

\begin{itemize}
  \item Generate simple tools through a compact specification
  \item \abc{} oriented
  \item Associate transformations with specific \abc{} elements, allowing a surgical processing
  \item Rich embedding mechanisms (using Perl for specific \abc{} transformations)
  \item Apply the identity function to not specified elements (default transformation)
  \item Processing guided by the music's internal structure
  \item Transform and manipulate the internal structure as it best suits the task at hand for a more
  efficient processing
\end{itemize}

\subsubsection*{Musical Information Representation}

A score's \ac{IR} must:

\begin{itemize}
  \item \textbf{Keep the original order of the score elements} \hfill\\
    This is an obvious goal since almost every task needs to know the exact order of a score to
    produce anything useful.
  \item \textbf{Hold sufficient musical information to rebuild the score as it was} \hfill\\
    This way a score can easily be outputted as it was originally.
  \item \textbf{Have different views of its structure} \hfill\\
    In order to have a thorough and efficient processing, the structure may be reorganized into one
    oriented to the part (for melodic tasks), to the time (for harmonic tasks) or to the source (for
    general tasks, mainly to be able to reconstruct the original \abc{}).
  \item \textbf{Facilitate the application of scripting} \hfill\\
    This means the \ac{IR} can be serialized into a structure that is easily evaluated by a language
    like Perl.
\end{itemize}

\subsubsection*{Musical Corpora}

In order to do statistical analysis there must be:

\begin{itemize}
  \item \textbf{Musical corpus comprised of musical scores} \hfill\\
    This corpus will be used as a source of data for the analysis as well as testing material.
  \item \textbf{Build tools for statistical calculation} \hfill\\
    Each tool will output some statistical information.
\end{itemize}

\subsubsection*{Musical Information Visualization}

There are always different forms for displaying musical information to a user. Be it a graph, a
drawing, some sort of symbol, the actual score or even simple text, the output must always transmit
knowledge to the user so that a conclusion can be taken from it. Thus a tool's output must:

\begin{itemize}
  \item \textbf{Have an appropriate format (textual, graphical, other)} \hfill\\
    So that the user can make the most out of the tool's results.
  \item \textbf{Be easy to comprehend} \hfill\\
    The results cannot be cryptic, otherwise the user will not understand.  
  \item \textbf{Reveal some feature of the music} \hfill\\
    If an analysis is made to a score then some sort of feature, hidden or not, must be revealed.
\end{itemize}
