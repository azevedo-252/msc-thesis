The representation of musical information is an area of research that has been receiving
contributions throughout time and it's not expected to be a consensus about a universal structure.

One of the most influent matters in making such representations is their final purpose. There are
different intentions, such as music rendering, play back, printing, music analysis, composition,
among others. The scope of this thesis includes only music rendering and analysis, therefore the
representation will have a well defined orientation which will not take into account additional
components that would benefit, for instance, composition tasks.

There are many models, data structures, paradigms, techniques, systems and theories proposed by many
authors~\cite{Brinkman1984,Buxton1978,Bilmes1992,Smaill1993,Wiggins1989,Dannenberg1993} and none can
be labeled as \emph{the perfect} representation, as there will never be a closed definition of
music and it is still difficult to represent all aspects of music.

% As will be explained in section \ref{sec:abc_dt}, this work presents a Perl module called \abcdt{},
% which can be viewed as a \ac{DSL} embedded in Perl. It processes some input information and returns it.
% The input information is parsed and an \ac{IR} is generated. That representation
% guides how the processing will be done. So, it is important to establish its structure, as it will
% determine how an \abc{} tune may be processed.

This dissertation's aim is to obtain a structure that allows the easy manipulation of music in a
computer. That structure should be compliant with a variety of tasks regarding music rendering and
analysis, such as, representing pitch and determining intervals from them, and obtaining all musical
elements that occur in a specific musical moment. Most importantly, it must allow the reconstruction
of the original \abc{}, in other words, it must contain all the original information including the
order of each element.

Next, a discussion on two topics is made: the pros and cons of sequential vs. hierarchical
representations and melody vs. harmony representations. The latter is mentioned again in section
\ref{sec:parse}.

\subsection{Sequential vs. Hierarchical}

The most used structures for music representation are the sequential and hierarchical structures.

\subsubsection*{Sequential}

In the beginning, computer music systems represented music as a simple sequence of notes. It was a
simple approach, thus making it difficult to encode structural relationships between notes, such as
enveloping a group of notes in order to apply some kind of property.

For instance, \midi{}~\cite{midi:Online} has no mechanisms for describing new structural
relationships.  However, \midi{} has a number of predefined structures. There are 16 channels, which
effectively form 16 groups of notes. Each group has a set of controllers such as volume and
pitch-bend. This gives \midi{} a limited 2-level structure.

The sequential structure refers to a sequence of any kind of musical component, usually indexed by
ordinal position rather than time. For instance, a musical element, such as a measure bar, is
referred as the fourth element rather than an element at time 5.7 seconds or at musical offset 3,
corresponding to the length of 3 quarter notes.

Many groupings of interest in music are likely to exhibit this property of strict ordering - most
melodies, for example, are monophonic. An analysis task's efficiency may improve when dealing with
this kind of structure.

\subsubsection*{Hierarchical}

It is widely accepted that music is best described at higher levels in terms of some sort of
hierarchical structure~\cite{balaban1987music}. This kind of structure has the benefit of isolating
different components of the score, therefore allowing transformations, such as tempo or pitch, to be
applied to each of them individually. It also represents a set of instructions for how to put the
score back together, hence allowing to reassemble it as it was.

Musical events can spread behavior to other events through the binary relation \textit{part-of},
which denotes relations like "measures part-of phrase". They can also inherit behavior and
characteristics from other events through the \textit{is-a} relation, which designates relations
like "a dominant chord being a special kind of seventh chord"~\cite{Honing1993}.

There are other kinds of relations that are needed as well in a hierarchical representation.
Honing\cite{Honing1993} suggests:

\begin{itemize}
  \item Binary:
    \begin{itemize}
      \item \textbf{associative:}
        e.g.: "A theme with its variations"
      \item \textbf{functional:}
        e.g.: "The function of a particular chord in a scale"
      \item \textbf{referential:}
        e.g.: "A theme referring to a previously presented or already known motif"
    \end{itemize}
  \item N-ary:
    These relations can structure more complex types of relation.\\
    e.g.: "the dependency of a certain chord on scale, mode and the context in which it is used is a
    ternary relation"
\end{itemize}

A single hierarchy scheme is not enough because music frequently contains multiple hierarchies, for
instance, a sequence of notes can belong simultaneously to a phrase marking and a to section (like a
movement). So the need of a multilevel hierarchy appears. There are some other possible hierarchies:
voices, sections (movement, measure), phrases, and chords, all of which are ways of
grouping and structuring music.

A few representations have been proposed~\cite{Dannenberg1990,Brinkman1984} that support multiple
hierarchies through named links relating musical events and through instances of hierarchies. And
others where tags are assigned to events in order to designate grouping, such as, all notes under a
slur.

\subsection{Melody vs. Harmony}

In polyphonic music there are materials besides melody that are combined in a score: rhythm and
harmony. Those three (melody, rhythm and harmony) determine the global quality of a
score~\cite{benward2003music} and their combination is usually called a texture. When there's only
one voice (melody), accompanied or not by chords, it is called monophony, but when there's two or
more independent voices, it is called polyphony.

The study of independent melodies is relatively simple compared to the analysis of polyphony. Each
voice moving through the horizontal dimension creates other effects by overlapping with notes in
other voices. The necessity for representing these vertical structures arises so that the harmonic
motion can be analysed.\\

Four suggestions of representations arise from this discussion: \partwise{}, \timewise{} and
\emph{hybrid} and \sourcewise{}.

\subsubsection{Part-wise}

The \partwise{} representation expresses a score by part (voice, instrument). So each part contains
many tuples (voice , \abcelements{}). Each \abcelement{} belongs to the specified voice. See example
\ref{ex:partwise}.

\begin{program}
  score $\rightarrow$ part*\\
  part $\rightarrow$ (voice, abc\_element*)\\
  \caption{\emph{Part-wise} representation}
  \label{ex:partwise}
\end{program}

This representation is best suited for melodic studies, considering that it is possible to directly
obtain all \abcelements{} belonging to a voice.

\subsubsection{Time-wise}

The \timewise{} representation expresses a score by time (musical moment), this is, by the offset of
the elapsed time. So each musical moment contains many parts. A part is a tuple (\emph{time offset}
, \emph{voice} , \abcelement{}). See example \ref{ex:timewise}.

\begin{program}
  score $\rightarrow$ musical\_moment*\\
  musical\_moment $\rightarrow$ part*\\
  part $\rightarrow$ (time\_offset, voice, abc\_element)
  \caption{\emph{Time-wise} representation}
  \label{ex:timewise}
\end{program}

This representation is best suited for harmonic studies, considering that it is possible to directly
obtain all musical events that occur in a specific moment in time.

MusicXML allows the representation of both time and part dimensions in two separate schemas. The
\partwise{} schema represents scores by part/instrument and the \timewise{} schema by time/measure.

\subsubsection{Hybrid}

The \emph{hybrid} representation derives from the need to solve an issue related with the
variability of a score's texture. This is, a score may have different densities of notes per part
and it is required that all events occurring at the same time are vertically aligned.

So, Brinkman~\cite{Brinkman1984} suggests a solution that uses a linked representation of a sparse
matrix. Each row of the latter references a part and each column the offset of the elapsed time,
which would enable traversing the score in any direction required (vertical or horizontal). Thus,
attaining a perception of the context of what's happening in a specific part, a feature that can't
be achieved when dealing with representations with only one dimension. Moreover, it makes the task
of score segmentation by part or time easier.

\subsubsection{Source-wise}

The \sourcewise{} representation expresses a score as it is parsed from the \abc{} file. So a score
is an ordered list of tuples (\abcelement{} , \context{}). The \context{} keeps record of an
\abcelement{}'s contextual information, this is, it keeps track of the current time offset, the
current voice, among other information (this \context{} is explained in more detail in section
\ref{sec:abcdt_rules}). See example \ref{ex:sourcewise}.

\begin{program}
  score $\rightarrow$ (abc\_element , context)*\\
  context $\rightarrow$ (current\_time, current\_voice, ...)
  \caption{\emph{Source-wise} representation}
  \label{ex:sourcewise}
\end{program}

This representation is best suited for rewriting purposes, considering that the actual order of
\abcelements{} is the same as in the original \abc{} file. It's also best for generic processing
where there's no need for a specific representation.


\subsection{Summary}

Two structures most commonly approached by researchers for representing music were discussed:
sequential and hierarchical. However, the decision of which structure type one should choose relies
on the purpose the \ac{IR} will have, such as, rendering of music, printing, music analysis,
composition, etc. This is, it relies in how many and what kind of questions are to be made to it.

A sequential structure may benefit certain tasks where a fast and simple traversal and/or a strict
ordering of its elements are required.

A hierarchical structure allows to isolate different components of a score and establish relations
between them. These are obvious advantages, although its traversal requires more advanced
techniques.\\

Regarding the horizontal and vertical dimensions of polyphonic music, four representations were
suggested. A \partwise{} representation is best suited for melodic studies, \timewise{} for harmonic
studies, \emph{hybrid} for both and \sourcewise{} for generic processing.

The most relevant disadvantage of the first three representations is that they don't maintain the
original order of elements. For instance, in \abc{}, it is common to write a part alternately with
other parts like (voice A, voice B, voice A, voice B). Meaning that a fragment of part A's music is
written first, followed by a fragment of part B, then another fragment from voice A and another from
voice B. When representing a score oriented to a vertical axis, the order in which each element is
written in the source file is lost, thus invalidating tasks like re-rendering \abc{} tunes.

% An assessment is taken after a, not so thorough, research on representations and their pros and
% cons: sequential and hierarchical structures are more suited to horizontal readings and tasks such
% as re-rendering an \abc{} tune, since they preserve the original order of the elements.

% Whereas, structures like sparse matrices grant both horizontal and vertical readings. Such
% structures provide a representation better suited to purposes requiring a quick access to vertical
% events on a score. Yet, it does not maintain the original order of elements. For instance, in
% \abc{}, it is common to write a part alternately with other parts like (voice A, voice B, voice A,
% voice B). Meaning that a fragment of part A's music is written first, followed by a fragment of part
% B, then another fragment from voice A and another from voice B. When representing a score oriented
% to a vertical axis, the order in which each element is written in the source file is lost, thus
% invalidating tasks like re-rendering \abc{} tunes.

A perfect representation would be one that is sufficiently generic and complete to be useful
in different analytic tasks in many styles of music~\cite{Honing1993}, like expressing common
abstract musical patterns.
