\documentclass[main.tex]{files}
\begin{document}

\chapter{Corpora}

To be able to calculate the difference between what is considered a pattern, assess what is expected,
calculate similarities between scores or generate statistics there must exist some example cases.

Those example cases are called corpus and in the case of this thesis it is a specific corpus - a
musical corpus - which contains a set of rich metadata of the musical scores. This corpus grants
the toolkit a mean to take advantage of the information calculated in the analysis of corpora by
enabling a richer combination of tools.

Another uses for the corpus include using them as testing material to the toolkit, for instance a
tool that validates the syntax needs either flawless examples and examples with deliberately typed
errors to guarantee that it works as it's supposed to.\\
Also using them to train systems that can learn from data. For instance, a system that is trained
with a set of scores to learn to identify a certain music style.

\section{Building corpora}

This phase, according to the existing literature on building corpora\cite{Atkins1992,
wynne2005developing}, consists on planning the whole process, getting permissions for gathering
scores and annotating them.

\subsection{Planning}

In this step decisions have to be made so that the remaining steps may take place:
\begin{itemize}
  \item Define the quantity of scores that will be added to the corpus;
  \item Select the scores that should be added;
  \item Define the intermediary formats and conventions to be used in the processing pipeline;
  \item Define if and what annotations should be included in the corpus;
  \item Define the formats it should be available in and how should the analysis tools interface
    with them.
\end{itemize}

\subsection{Getting permissions}

Scores will be gathered from public domain as they are not subject to copyright.

\subsection{Gathering scores}

Abc notation has become very popular since its introduction, and nowadays thousands of tunes -
scores, as in a music consisting of multiple voices, exist in a lesser number because the
features that allow the writing of polyphony were only added to the standard much later, yet
it's an ever growing culture - exist in electronic format. However a previous parsing and
reformatting might still be needed;

\subsection{Annotating scores}

In order to improve the usefulness of a corpus for a richer and more rigorous statistical analysis,
it might be subject to the process of annotation. It consists on applying some sort of structural
representation to act as a blue print of the original text and to provide additional interpretative
information.

Yet the use of annotations is not always the best approach. Susan Hunston, in
\textit{Corpora in Applied Linguistics}\cite{hunston2002corpora} says that the kind of questions
that are usually asked to a corpus rather than the questions that can be asked tend to be limited.
This happens because the categories used to annotate a corpus are frequently determined
before any analysis is done. Thus, as mentioned before, a careful planning has to be made.


\section{What can be analyzed} 

It's desired that a set of tools for statistical calculation is built. To make it possible a large
set of corpora (plural of corpus) must be built, so that statistical analysis and hypothesis
checking\footnote{Hypothesis testing is the use of statistics to determine the probability that a
given hypothesis is true.} can be done with it and from its results extract valuable information.

Some examples of what kind of analysis can be done with musical corpora:
\begin{itemize}
  \item Find sets of vertical patterns that occur in a large number of scores in the
    corpus\cite{Conklin2002};
  \item Find significant statistical differences between melodies of folk music from different
    countries\cite{chai2001folk};
  \item Display multiple excerpts from a collection of scores, such as all of the cadences, without
    human intervention and editing\cite{Knopke};
  \item Measure rhythmic similarity (the repetitive nature of the music) with manual annotations to
    the corpus\cite{antonopoulos2007music};
  \item Study norms of behavior for documenting relationships between accents and harmonic structure
    in common-practice music, and the role of notational variants in identifying scribes and
    composers\cite{Ariza};
  \item Identify trends and changes throughout a historical time period through cluster
    analysis\cite{albrechtemergence}.
\end{itemize}

\section{Existing Corpora} 

There are many existing musical corpus available in the Internet. A large corpus will be assembled
ranging from Abc corpus, to Midi, MusicXML and still LilyPond.
The corpus format may vary depending on what the tools can read and process, for instance, if midi
transformation is implemented then midi corpus have to exist as well. Presently Abc is the main
format therefore the initial focus will be in Abc corpus.

Here are some of the websites and packages from where they'll be gathered:
\begin{description}
  \item[http://abcnotation.com/browseTunes] \hfill \\
    Around 350,000 tunes available as Abc or Midi sound files;
  \item[http://thesession.org/] \hfill \\
    Around 11,000 tunes available as Abc or Midi sound files;
  \item[http://moinejf.free.fr/abc/index.html] \hfill \\
    Abc organ pieces;
  \item[http://www.classicalarchives.com] \hfill \\
    Around 14,000 Midi sound files;
  \item[http://abc.sourceforge.net/NMD/] \hfill \\
    Around 1000 Abc files;
  \item[Music21 corpus package] \hfill \\
    A collection of approximately 10,000 works provided with the toolkit, freely-distributable music
    for analysis and testing, including a complete collection of the Bach Chorales, numerous
    Beethoven String Quartets, and examples of Renaissance polyphony. The corpus includes Abc,
    MusicXML and Kern files.
  \item[http://wiki-score.org/] \hfill \\
    It is intended for publishing modern editions of unknown works buried in music archives in Abc.
  \item[http://www.mutopiaproject.org/] \hfill \\
    The Mutopia Project offers sheet music editions of classical music in LilyPond.
\end{description}

\section{Summary}

Musical corpus will be built according to the needs of the toolkit. The initial need is for a
toolkit that reads and processes Abc notation so the main focus will be to build an Abc corpus.

A careful planning on how to build the corpus plays an important role on determining the quality and
quantity of tasks that can use it. Such planning strongly affects the final results a tool can
produce.


\end{document}
