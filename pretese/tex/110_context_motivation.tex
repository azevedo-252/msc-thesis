\documentclass[main.tex]{subfiles}
\begin{document}

The main incentives that justify the need for writing such a document are presented in the following
sections.

\subsection{Textual Musical Notation}

All music needs to be written before read, comprehended or performed by any musician. To make it
possible, a notation system has been developed that provides musicians with the information
necessary to reproduce it as the composer wanted.\\
The notation consists on any system that represents audible music through written symbols. The use
of symbolic and abstract formats improves music reasoning, as it gives the composer a greater
freedom to express his music and provides easier readability to the performer.\\
As computers were introduced to the world of music, a variety of file formats and textual notations
began to emerge in order to store notation, such as, Abc, LilyPond or MusicXML.

\subsection{Textual Musical Notation Processing} 

More and more, musicology aided by computer is a reality which leads to an ever growing research in
some fields of music such as tonal analysis and composition. Likewise, a contribution in these
research fields is expected, which will provide tools for analyzing, detecting errors, restoring
scores and producing statistical results when applied on abc scores.\\
Composition is not included in the scope of this dissertation as it is a field rather complex and
that would not fit in the regular schedule. However as an interesting field of research as it is, it
would be a great feature to include in future work.

\subsection{Unix Philosophy}

\begin{quotation}
  \small\textit{Unix is simple. It just takes a genius to understand its simplicity.}
  \begin{flushright}
    Dennis Ritchie
  \end{flushright}
\end{quotation}

In the 1970s the \ac{SO} Unix was born and with it a new philosophy\cite{raymond2004art} based on a
principle that stated that creating robust and complex tools wasn't the right way, but instead
preferred tools of smaller scale, simple and efficient which tackle only one problem at a time and
whose results articulation is possible.

The main interface with the system is the command line where a user runs commands and programs. Thus
making the work method very powerful and flexible as it is possible to compound and execute commands
automatically.

Many Unix commands execute functions of reduced dimension and their input/output share the same data
type which allows the articulation of commands. Such philosophy can be easily applied to music
scores.

There are some Unix commands whose behavior can be adapted:
\begin{description}
  \item[grep:]
    Prints the lines of a file matching a pattern;\\
    It could print melodic sequences matching a pattern. The pattern could be a sequence of melodic
    intervals or rhythms;
  \item[diff:] 
    Compares files line by line;\\
    It could compare files voice by voice;
  \item[\ldots]
\end{description}

\subsection{Application Areas}

Some of the activities that could benefit from this toolkit are:
\begin{description}
  \item[Musical Wikis] \hfill \\
    Wikis that deal with the edition of musical scores. E.g.: Wiki::Score.
  \item[Cultural and cooperative volunteering] \hfill \\
    In environments like Wikis where the edition of documents happens in cooperation with many
    elements concurrently. E.g.: Wiki::Score.
  \item[Score transcription] \hfill \\
    Often errors occur while manually transcribing a score and those errors are not easily detected.
  \item[Musical analysis and composition] \hfill \\
    E.g.: through the detection of certain patterns in specific musical style it is possible to
    assess if a composition uses any feature of that style.
\end{description}

\end{document}
