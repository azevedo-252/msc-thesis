\paragraph{}

\abc{}~\cite{abcnotation:Online} é uma notação musical simples mas poderosa que permite a produção
de partituras completas e profissionais.

Atualmente, existe uma escassez de ferramentas genéricas para processamento de notação musical,
particularmente para \abc{}.

Esta dissertação apresenta o \abcdt{}, uma linguagem de domínio
específico~\cite{kosar2010comparing,kosar2008preliminary} baseada em regras (embutida em Perl),
projetada para simplificar a criação de ferramentas para processamento de \abc{}. Inpiradas na
filosofia \unix{}, essas ferramentas pretendem ser simples e composicionais à semelhança dos filtros
\unix{}.

A partir das regras do \abcdt{} obtém-se uma ferramenta para processamento de \abc{} cujo algoritmo
principal segue a arquitetura de um compilador tradicional, dessa forma consistindo em três fases:
\textbf{1)} parsing de \abc{} (baseado no parser do \abcmtops{}~\cite{abcm2ps:Online}), \textbf{2)}
transformação semântica de \abc{} (associada a atributos \abc{}) e \textbf{3)} geração de output (um
gerador definido pelo utilizador or fornecido pelo sistema).

Um conjunto de ferramentas para processamento de \abc{} foi desenvolvido utilizando o \abcdt{}.
Cada uma delas tem uma finalidade única, desde detetar erros, a auxiliar no estudo de música e até
imitar o comportamento de algumas ferramentas \unix{}. Estas têm o objetivo de serem provas de
conceito e ainda podem ser melhoradas, no entanto demonstram quão facilmente ferramentas compactas
para processamento de \abc{} podem ser criadas.

Um teste e avaliação foram realizados a uma das ferramentas criadas (\canonabc{}) com uma partitura
\abc{} real, o \texttt{Canon de Pachelbel}.
