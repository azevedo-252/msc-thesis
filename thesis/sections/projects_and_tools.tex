In this section some of the most relevant projects and tools being developed or used at the moment
are discussed\footnote{A more extensive list of \abc{} software may be consulted in
http://abcnotation.com/software\#linux}.

\begin{description}
  \item[\textbf{abcm2ps}]~\cite{abcm2ps:Online}
    A command line program which translates music written in \abc{} music notation into customary
    sheet music scores in PostScript or SVG format.

    It is based on \texttt{abc2ps} 1.2.5 and was developed mainly to print Baroque organ scores that
    have independent voices played on multiple keyboards and a pedal-board. The  program has since
    then been extended to support various other notation conventions in use for sheet music.
    Moreover, it is now one of the most complete \abc{} implementations.

    It is developed in C language and the author, an organist and programmer called Jean-François
    Moine, releases “stable” and “development” versions of his program. As of this
    writing\footnote{\today}, the stable release is 6.6.22 and the development release is 7.6.0.
    Since release 7.2.1, \abcmtops{} tries to follow the \abc{} standard version
    2.1\footnote{http://abcnotation.com/wiki/abc:standard:v2.1}.

  \item[\textbf{abc2midi}]~\cite{abc2midi:Online}
    A program that converts an \abc{} music notation file into a \midi{} file.

    It is part of the abcMIDI package, which includes other utility applications.

    The program was developed in C language by James Allwright in the early 1990s and has been
    supported by Seymour Shlien since 2003. It contains many features, such as expansion of guitar
    chords, drum accompaniment, and support for micro tones which do not exist in other packages.

  \item[\textbf{tclabc}]~\cite{tclabc:Online}
    A \emph{tcl}\footnote{Tcl is a scripting language created by John Ousterhout. It is commonly
    used for rapid prototyping, scripted applications, GUIs and testing. Tcl is used on embedded
    systems platforms, both in its full form and in several other small-footprint versions.}
    extension which permits ABC tunes parsing and editing.

    \begin{itemize}
      \item the \abc{} tunes are converted into an \ac{IR} suitable for many tcl operations, without
      losing the original tune information;
      \item most of the \abc{} specification is supported;
      \item the headers and tune symbols may be changed in many ways;
      \item transposition is done automatically when changing the key signature;
      \item bars may be automatically inserted;
      \item \midi{} files may be imported and exported;
      \item partial dump/include solves the selection copy/paste functions;
      \item \midi{} input and output are supported on many systems;
    \end{itemize}

  \item[\textbf{Music21}]~\cite{music21:Online}
    A Python-based toolkit for computer-aided musicology.

    Music21 is a set of tools for helping scholars and other active listeners answer questions about
    music quickly and simply.

    Music21 builds on preexisting frameworks and technologies such as Humdrum, MusicXML, MuseData,
    \midi{}, and LilyPond, but Music21 uses an object-oriented skeleton that makes it easier to
    handle complex data. At the same time, Music21 tries to keep its code clear and makes reusing
    existing code simple.

    Applications of this toolkit include computational musicology, music informations, musical
    example extraction and generation, music notation editing and scripting, and a wide variety of
    approaches to composition, both algorithmic and directly specified.

    It also has a large corpus of musical scores in many formats, including \abc{} and MusicXML.

  \item[\textbf{abctool}]~\cite{abctool:Online}
    A Python script that manipulates music files in \abc{} format.

    It's mostly useful for people working on the command line and/or editing \abc{} directly in an
    editor. It relies on external programs for certain tasks like converting into PostScript or
    transposing.

    Its main features are reading from standard input or file, outputting to standard output
    (PostScript, PDF or \midi{}), viewing (using \abcmtops{} and \texttt{gv}), transposing,
    translating chord names to Danish/German, and removing chords and fingerings.

    It is open source, developed by Atte André Jensen and released under GPL.

  \item[\textbf{Haskore}]~\cite{hudak1996haskore}
    Haskore is a set of Haskell modules for creating, analyzing and manipulating music.

    The formal approach used in this project is very elegant and powerful and is a very good
    studying resource. Nevertheless, when one wants to process existing \abc{} music, there are many
    details that don't fit in Haskore model like slurs, dynamics, microtones. In order to process
    them, those elements must be forgotten or drastic changes to the model must be introduced.

  \item[\textbf{EasyAbc}]~\cite{easyabc:Online}
    An open source \abc{} editor for Windows, OSX and Linux.

    It uses \abcmtops{} and \abctomidi{} and it has a rich features list. Most notably, it can
    import MusicXML files and export tunes in SVG format. It is published under the GNU Public
    License and was developed by Nils Liberg.

  \item[\textbf{abcpp Preprocessor}]~\cite{abcplus:Online}
    A simple yet powerful preprocessor designed for, but not limited to, \abc{} music files.

    It was written to overcome incompatibilities between \abc{} packages, and to facilitate writing
    portable and more readable \abc{} files. A preprocessor is a program that modifies a text file,
    according to commands contained in the file.

    It provides:
    \begin{itemize}
      \item conditional output;
      \item exclude or include parts of a piece according to specified conditions;
      \item define macros, i.e. symbols and sequences of customized commands;
      \item rename commands, symbols, and notes;
      \item include parts of other files.
    \end{itemize}

  \item[\textbf{ABCp}]~\cite{abcp:Online}
    A parser for the \abc{} music notation.

    It is a C library that interprets \abc{}. It is released as open source, under the terms of the
    BSD license, and may be used in both free and commercial software.

    ABCp has been designed with the following requirements in mind:
    \begin{itemize}
      \item to be able to handle the \abc{} 2.0 standard as well as previous standards and the
      extensions introduced by the most widely used tools (\abcmtops{}, abcMIDI, ...);
      \item to be fast;
      \item to be small: there must be a fair trade-off between size and functionalities;
      \item to be easily embeddable: no big restriction on the programming language to use;
      \item to be usable: no complex API or class hierarchy to remember.
    \end{itemize}

  \item[\textbf{Music::Abc::Archive}]~\cite{music_abc_archive:Online}
    A Perl module to parse \abc{} music archives.

    \abc{} music archives contain songs in the \abc{} format. This module encapsulates the \abc{}
    archive and individual songs so they may be managed more easily by Perl front-ends.

\end{description}

Some of the tools and projects presented were very relevant: \texttt{abctool} is a simple command
following \unix{}'s philosophy; \abctomidi{} and \abcmtops{} deal with processing real world \abc{}s,
but have specific purposes; \texttt{Music21} has similar goals and has a very powerful and complex
object oriented modules for music processing; \texttt{Haskore} is very flexible and elegant but
can't deal with real world \abc{} details.
