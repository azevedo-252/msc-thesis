\documentclass[main.tex]{files}
\begin{document}

\chapter{Internal Representation of Musical Information}
% Music is the art that intrigues researchers the most. A few puzzling questions arise from
% studying it like the meaning of music, we cannot retrieve meaning from it as we do from language
% and yet many listeners describe it as meaningful.
% 
% The metaphores related to music, there are some features of music that we cannot tell whether
% they are part of it or can only be described metaphorically, i.e. expressions like "music moves
% forward" shouldn't make sense when music is just a sequence of sounds.

The internal representation of musical information is an area of research that has been receiving
a lot of contributions throughout time and there will never be a consensus about the structure
it should have. One of the most influent matters to the making of such representation is the
final purpose it will have. There are different purposes like rendering of music, play back,
printing, music analysis, composition, among others.

The scope of this thesis includes only rendering and analysis of music, therefore the
representation will have a well defined orientation and a set of music properties will
automatically be discarded. There are many models, data structures, paradigms, techniques,
systems and theories proposed by many authors and none can be labeled as a "true" representation as
there will never be a closed definition of music and it is still difficult to represent all aspects
of music.

\section{Structure and Musical Entities Modeling}
In this section we will focus on a discussion about some kinds of structures and their benefits and
also different ways to represent a musical entity.

% \begin{itemize}
%   \item \textbf{Structure}
\subsubsection{Structure}

    In the beginning computer music systems represented music as a simple sequence of notes. This
    was a simple approach, thus making it difficult to encode structural relationships between
    notes such as enveloping a group of notes to apply some kind of property.

    It is widely accepted that music is best described at higher levels in terms of some sort
    of hierarchical structure [Balaban 88]. This structure should deal with events that can be
    either a single note or a whole score, applying operators to them in the same manner. Given
    the temporal nature of music, a score is defined as a sequence of those events allowing
    the existence of nested structures.

    This kind of structure has the benefit of isolating different components of the score
    therefore allowing for transformations to be applied to each of them. It also represents a
    set of instructions for how to put the score together, hence enabling to reassemble it as
    it was. This is a notion that many composition languages support.\\
    Musical events can spread behavior to other events through the binary relation \textit{part-of}
    relation which denotes relations like "measures part-of phrase" and inherit behavior and
    characteristics from other events through the \textit{is-a} relation which designates relations
    like "a dominant chord being a special kind of seventh chord".\cite{Honing1993}

    A single hierarchy scheme is not enough because music frequently contains multiple hierarchies.
    For instance a sequence of notes can belong simultaneously to a phrase marking and a section
    (like a movement), so the need of a multilevel hierarchy appears. There are some other possible
    hierarchies: voices, sections (movement, section, measure), phrases, and chords, all of which
    are ways of grouping and structuring music.

    A few representations have been proposed that support multiple hierarchies through named
    links relating musical events and through instances of hierarchies. Others where tags are
    assigned to events to designate grouping such as all notes under a slur.

  % \item \textbf{Musical Entities Modeling}
\subsubsection{Musical Entities Modeling}

    Musical entities, like a note, differ in the matter of being independent or serving as updates
    to some other entity, in other words being handled by some resource. Entities often have both
    characteristics at the same time.

    Some languages support the instance model where a copy or instance of a musical event that
    is independent of others is made. This model allows to take the redundancies that exist in
    musical structures and save memory space as well as simplify working with them.

    Other languages support resource-instance model\cite{Dannenberg1991} where events can be viewed
    as updates to some shared resource, like the notes of a melodic instrument that are not instances
    of the instrument tones, but updates to a shared instrument resource.

    Typically, a sequence of events is called a stream and it is subject to many operations like
    transposition. This notation is used by some music sequencers\footnote{A device or application
    software that can record, edit, or play back music, by handling note and performance information
    in several forms}.

    Components indexed by ordinal position rather than time are a concept related to streams
    and are called sequences. They are also used in a sequence-oriented compositional environment.
    
% \end{itemize}

\section{Melodic and Harmonic Structures} 

In polyphonic music there are materials besides melody that are combined in a score: rhythm
and harmony. Those three determine the global quality of a score[Benward \& Saker 2003] and their
combination is usually called a texture. When there's only one voice (melody) accompanied or not
by chords in a score it is called monophony, but when there's two or more independent voices
it is polyphony.

The study of independent melodies is relatively simple compared with the analysis
of polyphony. Each voice moving through the horizontal dimension creates other effects by
overlapping with notes in other voices. The necessity of representing this vertical structures
arises so that the harmonic motion can be analyzed.

Associated with this subject, one issue about the variability of the score's texture appears. A score
usually has different densities of notes per part for example. Since it is required that all
events that occur at the same time are vertically aligned a solution is required.

Brinkman\cite{Brinkman1984} suggests a solution that uses a linked representation of a sparse
matrix, where each row references a part and each column the onset of the elapsed time, which would
allow traversing the score in any direction (vertical and horizontal) required. Thus a perception of
the context of what's happening in a specific part appears, a feature that can't be achieved when
dealing with representations of only one dimension. Moreover it makes the task of score segmentation
by part and time easier.

\section{abcm2ps's approach} 
\label{sec:abcm2ps_approach}

The program abcm2ps was chosen to be the basis of a proof of concept. Mainly because it is one of
the most widely used programs for working with Abc, not just as a standalone software but as being
incorporated in many applications. The latter implies that it's not a piece of software that was
casually made. It was designed to process Abc in the best way possible therefore its quality is
acknowledged.\\
It is actively maintained and well documented which facilitates the analysis of the structures it
generates. Moreover, its author, Jeff Moine, was and still is a preponderant influence for the
evolution of the Abc notation and standard.\\
In the end, it seemed like a good starting point to do some preliminary experiments aiming to attain
a successful proof of concept. This proof of concept will be discussed thoroughly in chapter
\ref{chap:scripting_abc}.

As described before, abcm2ps converts Abc to music sheet. The conversion process uses a parser that
analyzes an Abc document and generates an internal representation.\\
This internal representation follows the sequential representation theory, in other words, each
symbol captured by the parser is simply appended to an ordered list. A symbol represents any kind of
component existing in the Abc notation, from header information - like the key or initial meter - to
a note, bar or a tuplet. After an initial analysis, its internal structure revealed itself as being
very rich with information from the original abc source, holding every single piece of information.

Since abcm2ps was designed to print abc notation its internal representation is not really suited
for music analysis or composition. Therefore it lacks all the benefits inherent to the other
representations described before like the sharing and inheriting of behavior and characteristics
between musical events.\\
Still it can be easily organized as a set of monophonic voices. This set might, for instance, be
used as starting point to describe relationships between vertical musical entities of a polyphonic
score.
%TODO talvez criar seccao sobre verticalidade. de como se pode juntar as varias vistas no processamento. talvez criar um funcao de verticalizacao no processador, ou outra coisa...

As the aim is for a toolkit built with scripts, the sequential structure reveals itself as an
appropriate structure that meet our requirements. The sequence of symbols the structure provides can
easily be mapped to an array or a hash. These data types are part of the common, yet powerful, data
types of a scripting language like Perl.

\section{Summary} 

The internal representation's types of structures were discussed and it showed that there are mainly
two types that are most commonly approached by researchers: sequential and hierarchic. However the
decision of what type of structure one should choose relies on the purpose the internal
representation will have: rendering of music, printing, music analysis, composition, etc.\\

Two kinds of binary relations inherent to a hierarchy were described: part-of and is-a but other
kinds of relations are needed as well. Honing\cite{Honing1993} suggests other relations:
\begin{itemize}
  \item Binary:
    \begin{itemize}
      \item \textbf{associative:}
        e.g.: "A theme with its variations"
      \item \textbf{functional:}
        e.g.: "The function of a particular chord in a scale"
      \item \textbf{referential:}
        e.g.: "A theme referring to a previously presented or already known motif"
    \end{itemize}
  \item N-ary:
    These relations can structure more complex types of relation.\\
    e.g.: "the dependency of a certain chord on scale, mode and the context in which it is used is a
    ternary relation"
\end{itemize}

Related to music analysis and composition purposes and regarding the horizontal and vertical
dimensions of polyphonic music, a solution to enable the harmonic analysis of a score was suggested.
A perfect representation would be one that was sufficiently general and complete to be useful in
many different analytic tasks in many styles of music\cite{Honing1993} like expressing common
abstract musical patterns.\\

An assessment that is taken after a not so thorough research on representations and their pros and
cons is that sequential and hierarchical structures are more suited to horizontal readings and tasks
such as re-rendering the Abc text since they preserve the original order of an Abc score's
symbols.\\
Whereas structures like sparse matrices allow both horizontal and vertical readings. That fact
provides a representation more suited to purposes that require a way of accessing to vertical events
on a score. Yet, it does not maintain the original order of symbols, for instance, in Abc it is
common to write a voice's music alternately with other voices like (voice A, voice B, voice A, voice B),
meaning that a part of voice A's music is written first, followed by that of voice B and after that
voice A and voice B again. When representing a score oriented to a vertical axis the order of events
is lost, thus preventing tasks like re-rendering Abc text.

Summarizing, the completeness of a structure depends on what purpose it is intended for and the
ability to fulfill all its tasks.\\

Different ways of modeling musical entities were discussed and two models were highlighted:
instance and resource-instance models.\\

At last the internal representation of abcm2ps was presented but not thoroughly described as it
will be in chapter \ref{chap:scripting_abc}.\\

Music has a lot to offer for the study of representations. Music contains complex structures of many
interrelated dimensions and since music and its representations are continuously evolving an ongoing
research on these areas is required.

\end{document}
