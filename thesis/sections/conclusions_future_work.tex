% This chapter presents a summary of the work performed in the context of this dissertation,
% establishing some conclusions about some .
%
% Additionally, future work possibilities are discussed, including .

\section{Conclusions}

The \unix{} philosophy and its simple and successful ideas were essential to the conception of this
dissertation:

\begin{itemize}
  \item The concept of simple and compact tools/commands that solve problems of small complexity and
  can be articulated with others was adopted;
  \item The universal type (the text stream) suggested \abc{} as the obvious universal type since it
  is text as well;
  \item The creation of the language \emph{C} to help developing \unix{} inspired the creation of a
  language as well (a \ac{DSL}) called \abcdt{}.
\end{itemize}

So, the natural course of events is to map some of the existing \unix{} tools into \abc{} ones like
\emph{cat}, \emph{paste} and \emph{wc}.\\


The strategy of creating a robust parser for a reflexive language (Perl) is obtained in 3 steps: 1)
searching the best tool that processes what is wanted; 2) isolating its parser; 3) adding a function
that traverses the parser's generated \ac{IR} and serializes it in order to be evaluated.

The \ac{IR} used must be complete enough to enable the application of many different analytic tasks.
However, that fact doesn't invalidate an approach that starts by generating a sequential and
\sourcewise{} structure and then transforming it into something that suits more complex demands.

Reusing \abcmtops{}' parser was very important to help guarantee this work's quality, coverage and
developing time. The generated \ac{IR} is \sourcewise{} which, along with the calculation of the
current \context{} (consult section \ref{sec:abcdt_rules}), allows performing a
representation-agnostic processing.\\


With the \ac{DSL} \abcdt{} there is a considerable simplification of the process of creating an
\abcpt{} considering the following features:

\begin{itemize}
  \item It's not necessary to specify what doesn't need to be transformed (default functions);
  \item A transformation specification is rule-based which facilitates its writing;
  \item There's a set of rich \actuators{} which allows to precisely select a specific point to
  transform.
\end{itemize}

Using a structured processing of \abc{} allows an \abcpt{} to be described in an effective and
compact way.  Processing a complex structure is possible if divided into smaller parts, i.e.,
applying many surgical transformations to its parts (according to its structure) and composing each
individual result into a single result.

A rule based processor (\abcdt{}'s \dt{} function) makes it possible to write very compact and
effective tools. Most of the processing is done by default, i.e. the user only needs to specify what
needs to be transformed. The default function is the identity function, \emph{toabc()}.

Using Perl as the language embedded into \abcdt{} provides a rich environment to allow easy
processing of text. Furthermore, through the use of data structures, like hashes, the user has
bigger expressive power to specify transformations.\\


Currently, there are tools that process \abc{} with specific purposes as well as big software
packages that integrate a lot of features, however there's always the need to process music, this
is, making custom modifications to the original \abc{}, producing some sort of information,
integrating existing tools, etc... That is the main reason for creating an \ac{OS} comprised of
simple tools for generic \abc{} processing which can be composed with each other, as well as a
versatile environment to create new tools through a compact \ac{DSL} embedded in Perl.

\section{Future Work}

As expected, there are many parts of this work that can be improved and extended. In this section,
only a few are going to be described.

\subsection*{Internal Representation}

A \partwise{} \ac{IR} is more suited to melody processing, while a \timewise{} \ac{IR} is more
suited to harmony processing and a \sourcewise{} to a general, robust processing. These
"\emph{views}" of the \ac{IR} provide different spacial perspectives over a score. Each one of them
covers different aspects of music and can even hide those less relevant.

So there will be a mechanism that, during the \ac{IR}'s traversal, recalculates the structure's
orientation when needed in order to provide different notions of the spacial context.

Consequently, a more thorough investigation on data structures and music representations is required
in order to obtain a set of \ac{IR}s capable of answering to very specific tasks in the best way
possible.

\subsection*{Musical Corpora}

Due to time limitations, no work has been done towards this area of research, even though it's one
of the most interesting areas that can bring a lot of useful features.

Building an \abc{} corpus to serve as testing material for the toolkit and also to train systems
that learn from data is still a main goal.

A set of statistical models is planned to be developed over \abc{} corpora in order to produce
features like automatic classification of scores by genre, time period, author, ...

The composition field has been, from the beginning of this dissertation's writing, postponed due to
its complexity. However, it's definitely an area of research worth working in, therefore one of the
features that will be developed is the automatic generation of music. For instance, training a
program to generate a score that follows a certain style, author, or other characteristics
associated, through the use of models, like \emph{Hidden Markov}~\cite{chai2001folk,Lo2010}. Also,
techniques of text mining~\cite{Tan1999} will be used to extract interesting and non-trivial
patterns from text documents.

\subsection*{ABC::DT}

\abcdt{} is far from being complete, it's actually in constant development. Still, some
improvements, extensions, even fixes are to be done in the near future.

The parser used - \abcmtops{}' - has its own bugs that the author has identified but not fixed yet,
as well as bugs that were identified during \abcdt{}'s development. So, whenever possible, any bug
found will be reported to the author in order to help to the parser's and the tools' it is used in -
\abcmtops{} and \tclabc{} - improvement.\\


The set of available \actuators{} will be expanded to accept even more detailed queries.

Another approach to \actuator{} specification will be devised. What's intended is a richer syntax
for identifying \abc{} elements, one that uses path expressions to navigate through an \abc{} tune
and has a set of standard functions to help selecting \abc{} elements. This approach is very similar
to \emph{XPath}'s\cite{XPath:Online}.

More default functions will be added to \abcdt{}'s API.

The identity function - \toabc{} - is not perfect and still needs some improvements.

\subsection*{Toolkit}
\paragraph{}
More \unix{}-like tools will be developed, such as:
\begin{itemize}
  \item \emph{grep\_abc}\\
    Prints melodic lines that match a pattern
  \item \emph{diff\_abc}\\
    Compares scores, for instance, voice by voice
  \item \emph{cut\_abc}, \emph{sed\_abc}, \emph{head\_abc}, \emph{tail\_abc}, \ldots
\end{itemize}

Other tools are planned to be developed:
\begin{itemize}
  \item \emph{transpose\_abc}\\
    Transposes a score up or down by an interval. There are already some tools that perform this
    task, such as \texttt{abc2abc}~\cite{abc2abc:Online} and the Perl script
    \texttt{transpose\_abc.pl}~\cite{transposeabc:Online}.

    It's desired that this feature is available as a polymorphic \abcdt{}'s function, i.e., it may
    be applied to the \emph{note} element, as well to the \emph{key} element and also an
    \emph{accompaniment chord}.
  \item \emph{fugue\_abc}\\
    Similar to \canonabc{}, only that each voice may change the pitch of the original theme. The
    fugue form, is not simple and has more details to its structure, however this tool may serve as
    an initial structure to build one.
\end{itemize}

For the existing tools a few improvements will be done as well:

\begin{description}
  \item[wc\_abc] \hfill
    \begin{itemize}
      \item Add more counts
      \item Provide options to select specific counts from the output
    \end{itemize}
  \item[canon\_abc] \hfill
    \begin{itemize}
      \item Add the option to assign a name to a voice
      \item Add the option to assign a \midi{} instrument to a voice
    \end{itemize}
  \item[find\_chords\_abc] \hfill
    \begin{itemize}
      \item Add more chord formations to find
    \end{itemize}
  \item[detect\_errors\_abc] \hfill
    \begin{itemize}
      \item Add more errors to detect
      \item Error fixing
        \begin{itemize}
          \item Incomplete measures $\Rightarrow$ Add Rests
          \item Overflowing measures $\Rightarrow$ For example, comment the exceeding elements and
          annotate a FIXME
          \item Final bar $\Rightarrow$ Add bar
        \end{itemize}
    \end{itemize}
\end{description}
