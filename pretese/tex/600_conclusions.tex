\documentclass[main.tex]{subfiles}
\begin{document}

\chapter{Conclusions}

%tb é permitido por duvidas do que sera feito

This document addressed the topic of scripting on textual musical notation and an analysis was made
of the components required to support its implementation.\\
A proof of concept was demonstrated to prove that a certain approach to apply scripting on abc
scores had the potential of being used as the basis for the toolkit.\\
Moreover a list of known musical notations and some of the most relevant projects and tools were
presented.

A careful planning on how to build the corpus plays an important role on determining the quality and
quantity of tasks that can use it. Such planning strongly affects the final results a tool can
produce.

There are mainly two types of internal representation's structures that are most commonly approached
by researchers: sequential and hierarchic. To meet the established requirements for the toolkit, the
representation used must be complete enough to allow the application of many different analytic
tasks. However that fact does not invalidate an approach that starts by generating a sequential
structure and from it generate something more suited to more complex uses. Furthermore the final
goal is a toolkit built with scripts, hence the sequential structure reveals itself as an
appropriate structure to apply scripting skills. 

In order to demonstrate that it is possible to process, transform and generate Abc documents along
with statistics through a simple algorithm a proof of concept was devised. It meant to prove that by
tackling the parts of an elaborate problem, the complexity of building a tool that solves such
problem may be mitigated.


\section{Research Planning}

The next step will be resuming the development of the processor that was presented in the proof of
concept. At the same time verify that the parser used is robust. This includes guaranteeing that a
future change to, for instance, abcm2ps's parser is easily adapted to the scripts and other programs
created that depended on it. In other words, all developed components must be capable of handling
subtle changes to third party parsers.

Other tasks include:
\begin{itemize}
  \item Creating a set of tests that verify the correctness of the tools developed;
  \item Creating the Unix-like tools for music;
  \item Building Abc corpora;
  \item Developing statistical tools for music;
  \item Developing musical analysis tools;
  \item Writing of the dissertation.
\end{itemize}

\end{document}
