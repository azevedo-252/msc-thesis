\documentclass[main.tex]{subfiles}
\begin{document}

\chapter{Introduction}

In this chapter the purpose and goals of this document will be stated, describing what are the main
concerns and motivations for the development of such work.

\section{Context and Motivation}
\subfile{tex/110_context_motivation}

\section{Project Overview}
\subfile{tex/120_project_overview}

\section{Document Structure}
\subfile{tex/130_document_structure}


\section{Summary}

A toolkit for the processing and analysis of abc scores was presented as the solution to a problem
arisen in the context of the Wiki::Score - a Wiki for cooperative edition of Abc scores. This
problem revolved around the fact that the manual edition of scores could originate errors in the
score transcribed.

So the toolkit proposes a solution by providing a set of tools that individually solve a single
problem, like correcting a word that was misspelled, and whose output can serve as the input to
another tool. The approach used by this solution follows the Unix philosophy.

Additionally, an analytic feature is given by the toolkit by granting a set of tools that allows the
musical and statistical analysis of a score through the use of musical corpus. This corpus will be
comprised of Abc musical scores and can contain annotations that might help the analysis process.

\end{document}
