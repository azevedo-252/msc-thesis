\documentclass[main.tex]{subfiles}

\begin{document}

\pdfbookmark{Resumo}{resumo}
\chapter*{Resumo}

Presentemente, plataformas cooperativas para edição de partituras musicais, como a Wiki::Score que
utiliza a notação Abc, não têm à sua disposição utilitários de avaliação e deteção de erros, nem
ferramentas que auxiliem a musicologia. Esta carência impede os utilizadores de tirarem o melhor
partido dessas plataformas e proporciona um sentimento de limitação na composição e transcrição de
partituras.

Para colmatar estas falhas, e adotando a filosofia Unix, criar-se-á um toolkit, em que cada
ferramenta tratará um problema individualmente como a deteção de erros sintáticos/léxicos.

Componentes de natureza musicológica como a análise tonal ou a deteção de
padrões farão parte do toolkit. Desta forma, é necessária a construção de corpora de obras musicais. A análise estatística do mesmo
permitirá a extração de informação valiosa que poderá ser utilizada por uma ferramenta ou exibida ao
utilizador num formato específico.

Este toolkit compreende principalmente tarefas dedicadas à análise e exibição de música. Deste
modo, este necessita de um parser robusto para que haja uma recolha exaustiva de toda a
informação necessária para realizar essas tarefas. Para além disso, o toolkit permite a
aplicação de scripting na música baseado numa filosofia que proporciona o ambiente apropriado para o
processamento sistemático e de ordem superior de música.

\newpage 

\pdfbookmark{Abstract}{abstract} 
\chapter*{Abstract}

Nowadays, cooperative platforms for musical scores edition, like Wiki::Score which uses the Abc
notation, don't have available error evaluation and detection utilities, nor tools that support
musicology. This shortage of utilities prevents users of taking advantage of these platforms and
provides a sense of limitation in the composition and transcription of scores.

To amend these shortcomings and by embracing the Unix philosophy, a toolkit will be built in which
each tool will deal with a problem individually such as the detection of syntactical/lexical
errors. 

Features of musicological nature like tonal analysis or pattern detection will make part of
the toolkit. Thus building a corpora of musical scores with rich metadata becomes an
essential requirement. The statistical analysis of the latter will allow the extraction of valuable
information that can be used by a tool or displayed to the user in a specific format.

This toolkit comprises mainly tasks intended for the analysis and rendering of music.
Therefore it needs a robust parser in order to thoroughly gather all the information required to
fulfill those tasks. Moreover the toolkit allows the application of scripting on music based on
a philosophy that provides the appropriate environment for a systematic and higher-order processing
of music.

\newpage 

\pagenumbering{arabic}

\end{document}
