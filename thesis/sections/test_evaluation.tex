This chapter's goal is to help measure and analyze the behavior of \abcpt{}s created with \abcdt{}
with real world tasks and to support some claims that have been made throughout this dissertation. A
thorough evaluation was not possible mainly because of time limitations, although it's planned to
happen in the near future.

The \abcpt{} being evaluated is \canonabc{} which will process a real world \abc{} score -
\emph{Pachelbel's Canon}, a canon belonging Pachelbel's \emph{Canon and Gigue for 3 violins and
basso continuo}.

The melodic part (here called \textbf{violini.abc}) can be seen in appendix \ref{sec:pcanon_melody}
and the accompaniment (here called \textbf{basso.abc}) in \ref{sec:pcanon_accomp}.

Listing \ref{lst:canonabc_pach} shows how \canonabc{} is called.\\

\begin{lstlisting}[caption={\canonabc{} for Pachelbel's Canon},label={lst:canonabc_pach},captionpos=t,abovecaptionskip=-\medskipamount]
canon_abc violini.abc+8 violini.abc+16 violini.abc+24 basso.abc++
\end{lstlisting}

The output generated is shown in appendix \ref{sec:pcanon}.\\

Using \wcabc{} on the generated canon (see listing \ref{lst:canon_wc}, the pitch counts were removed
since they are not needed for this explanation), there are 4 voices, each with 168 measures. The
individual melody (\textbf{violini.abc}) has 144 measures (see listing \ref{lst:melody_wc}, pitch
counts removed) and considering that the biggest number of \measurerests{} that are inserted to each
melody is 24 (as can be seen on \ref{lst:canonabc_pach} and \ref{fig:pcanon_scheme}), the number of
expected measures per voice in the canon is 168 (144+24), which matches the count produced in
listing \ref{lst:canon_wc}.\\

\lstinputlisting[caption={\wcabc{} on Pachelbel's Canon},label={lst:canon_wc},captionpos=t,abovecaptionskip=-\medskipamount]{misc/canonwc.tex}

\lstinputlisting[caption={\wcabc{} on Pachelbel's Canon Melody},label={lst:melody_wc},captionpos=t,abovecaptionskip=-\medskipamount]{misc/melodywc.tex}

Furthermore, the number of notes in each of the first three voices (melody parts) is the same as in
the original melody as expected considering that the only thing that \canonabc{} inserts into
melodic parts is \measurerests{}.

The accompaniment part (\textbf{basso.abc}) has 8 notes and 8 measures (see listing
\ref{lst:accomp_wc}, pitch counts removed). Since it is repeated until it has the same number of
measures as the melody, the expected result is 168 notes in 168 measures which matches the count
produced in listing \ref{lst:canon_wc}.\\

\lstinputlisting[caption={\wcabc{} on Pachelbel's Canon Accompaniment},label={lst:accomp_wc},captionpos=t,abovecaptionskip=-\medskipamount]{misc/accompwc.tex}

The execution time for this test's case rounds the 2.3 seconds (see table
\ref{tab:execution_times}). It's quite acceptable since it has 4 voices and 168 measures each which
translates into a lot of external calls to \catabc{} and \pasteabc{} in order to gradually build
each individual voice and finally put them all together.

\begin{center}
  \begin{table}[h]
    \begin{tabular}{|p{10cm}|p{4cm}|}
      \hline
      Tool & Execution Time\footnotemark[1]\\
      \hline
      \hline
      \canonabc{} violini.abc+8 ... ... ... > canon.abc & ~2.3 s
      \\
      \hline

      \hline
      \wcabc{} canon.abc & ~0.4 s
      \\
      \hline
    \end{tabular}
    \caption{Execution times}
    \label{tab:execution_times}
  \end{table}
\end{center}

\footnotetext[1]{The times presented were measured in a 5 year old laptop with the following
features: Processor: 2x Intel(R) Core(TM)2 Duo CPU T9400 \@ 2.53GHz; Memory: 3GB; Operating System:
Linux Mint 13 Maya}

The tool itself wasn't very hard to complete. It has around 85 lines of Perl code (not counting
empty lines), excluding \catabc{} and \pasteabc{} and it was quite quick to put together from the
moment the algorithm was designed.

All in all, this test with a real \abc{} score, even though not being an exhaustive one,
demonstrates that generating scores with many measures, notes and with a few minutes of length is
viable. This operation doesn't involve complex calculations, however there are several traversals to
the \abc{} structure as well as a few auxiliary calculations which affect the overall performance.
In the end it produces the expected result proving that an \abcpt{} built using \abcdt{} can deal
with real \abc{} and not with just some controlled testing tunes.
